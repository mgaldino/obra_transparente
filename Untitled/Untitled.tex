% AER-Article.tex for AEA last revised 22 June 2011
\documentclass[]{AEA}

% The mathtime package uses a Times font instead of Computer Modern.
% Uncomment the line below if you wish to use the mathtime package:
%\usepackage[cmbold]{mathtime}
% Note that miktex, by default, configures the mathtime package to use commercial fonts
% which you may not have. If you would like to use mathtime but you are seeing error
% messages about missing fonts (mtex.pfb, mtsy.pfb, or rmtmi.pfb) then please see
% the technical support document at http://www.aeaweb.org/templates/technical_support.pdf
% for instructions on fixing this problem.

% Note: you may use either harvard or natbib (but not both) to provide a wider
% variety of citation commands than latex supports natively. See below.

% Uncomment the next line to use the natbib package with bibtex
\usepackage{natbib}

% Uncomment the next line to use the harvard package with bibtex
%\usepackage[abbr]{harvard}

% This command determines the leading (vertical space between lines) in draft mode
% with 1.5 corresponding to "double" spacing.
\draftSpacing{1.5}


% tightlist command for lists without linebreak
\providecommand{\tightlist}{%
  \setlength{\itemsep}{0pt}\setlength{\parskip}{0pt}}


% Pandoc citation processing
\newlength{\cslhangindent}
\setlength{\cslhangindent}{1.5em}
\newlength{\csllabelwidth}
\setlength{\csllabelwidth}{3em}
\newlength{\cslentryspacingunit} % times entry-spacing
\setlength{\cslentryspacingunit}{\parskip}
% for Pandoc 2.8 to 2.10.1
\newenvironment{cslreferences}%
  {}%
  {\par}
% For Pandoc 2.11+
\newenvironment{CSLReferences}[2] % #1 hanging-ident, #2 entry spacing
 {% don't indent paragraphs
  \setlength{\parindent}{0pt}
  % turn on hanging indent if param 1 is 1
  \ifodd #1
  \let\oldpar\par
  \def\par{\hangindent=\cslhangindent\oldpar}
  \fi
  % set entry spacing
  \setlength{\parskip}{#2\cslentryspacingunit}
 }%
 {}
\usepackage{calc}
\newcommand{\CSLBlock}[1]{#1\hfill\break}
\newcommand{\CSLLeftMargin}[1]{\parbox[t]{\csllabelwidth}{#1}}
\newcommand{\CSLRightInline}[1]{\parbox[t]{\linewidth - \csllabelwidth}{#1}\break}
\newcommand{\CSLIndent}[1]{\hspace{\cslhangindent}#1}

\usepackage{booktabs}
\usepackage{caption}
\usepackage{longtable}
\usepackage{colortbl}
\usepackage{array}

\usepackage{hyperref}

\begin{document}


\title{The Civil Society Organizations effect: A mixed-Methods analysis
of bottom-up approaches in Brazilian public policy.}
\shortTitle{The Civil Society Organizations effect}
% \author{Author1 and Author2\thanks{Surname1: affiliation1, address1, email1.
% Surname2: affiliation2, address2, email2. Acknowledgements}}


\author{
  Manoel Galdino\\
  Bianca Vaz Mondo\\
  Juliana Mari Sakai\\
  Natália Paiva\thanks{
  Galdino: Universidade de São Paulo,
Brazil, \href{mailto:mgaldino@usp.br}{mgaldino@usp.br}.
  Mondo: Government Transparency Institute - Germany, \href{mailto:}{}.
  Sakai: Transparência Brasil - Brazil, \href{mailto:}{}.
  Paiva: Alandar Consulting - Germany, \href{mailto:}{}.
  We would like to thank all participants of project Obra Transparente,
  Vanda Medeiros and the team of Transparência Brasil who worked in the
  project or in evaluations of the project. The project was funded by an
  UNDEF grant, which we also thank.
}
}

\date{\today}
\pubMonth{11}
\pubYear{2023}
\pubVolume{}
\pubIssue{}
\JEL{}
\Keywords{accountability, Civil Society Organizations, mixed
methods, community-driven development, public policy.}

\begin{abstract}
This paper examines the effect of bottom-up accountability on public
service delivery. We differentiate between information-driven
interventions and the mobilization and monitoring efforts of organized
Civil Society Organizations (CSOs), and argue that the latter type of
interventions can can drive significant policy change. The study
evaluates the effectiveness of the Obra Transparente project by
Brazilian NGO Transparência Brasil, engaging 21 local CSOs in South and
Southeast Brazil. Our findings emphasize the importance of sustained,
coordinated efforts by socially embedded CSOS. These efforts, involving
direct engagement with municipal officials and ensuring that their
complains cannot be ignored lead to more substantial outcomes when
compared with citizens information-driven.
\end{abstract}


\maketitle

\hypertarget{introduction}{%
\subsection{Introduction}\label{introduction}}

Accountability is a crucial element in democratic systems, ensuring the
efficient delivery of public services (Besley and Ghatak 2003; Cameron
2004; O'donnell 1998). Initially defined as the process of holding
authorities responsible for their actions (Mulgan 2000; O'Loughlin
1990), the literature has since broadened to include various oversight
mechanisms for politicians and the bureaucracy. Some authors focus on
formal institutions (O'donnell 2003; Kenney 2003), known as horizontal
accountability, while others examine how politicians respond to voters,
the media, and civil society organizations (Bertelli and Van Ryzin 2020;
Relly 2012; Paul 1992), referred to as vertical accountability.

Recent studies have highlighted the role of ``bottom-up accountability''
(also known as social accountability, Fox, 2015), in which citizens
receive information about public policies and use it for social control.
The pioneering study `Power to the People' (Björkman and Svensson 2009),
which demonstrated the impact of local community engagement in
monitoring the provision of health services, inspired subsequent
literature that tested similar interventions in different contexts.

The mechanism would works like this: Constituents have firsthand
information about the outcomes of local policies, giving citizens
incentives to combat corruption that directly affects them. Since
policymakers are sensitive to social feedback from their own communities
(Serra 2012), bottom-up accountability could influence local public
policies

However, a series of recent studies have raised doubts about the
effectiveness of ``bottom-up accountability'' strategies based on
interventions aimed at increasing the monitoring capacity of ordinary
citizens regarding public policies (Freire, Galdino, and Mignozzetti
2020; Raffler, Posner, and Parkerson 2019; Fox 2015).

Most of the interventions are tactical, in the sense that they involve
providing information to non-organized citizens through technological
means, such as sending SMS messages or delivering information via mobile
apps. One key advantage of such interventions is that, if successful,
they could rapidly reach thousands of people at a low cost, potentially
having a significant impact when scaled to an entire country or even
multiple countries.

The disappointing results have cast doubts on the effectiveness of the
bottom-up accountability strategy for improving the delivery of public
services, reducing corruption, and enhancing government efficiency. Some
authors (Fox 2015; Ankamah 2019) have suggested that social
accountability without ``teeth,'' defined as the government's capacity
to respond to citizens' voices, has limited impact.

While it has been consistently shown that tactical interventions have,
at best, a small, limited impact, it doesn't mean that voice without
``teeth'' is an ineffective strategy. The argument in this study is that
not all types of interventions that promote citizen's voices are the
same.

We argue that we need to differentiate between information-led tactical
interventions and active mobilization and monitoring by Civil Society
Organizations (CSOs). By their very nature, established CSOs are
embedded in society and politics and can have a substantial impact on
policy change (Gong and Xiao 2017; Joshi and Houtzager 2012).

The paper argues that a missing feature in most of the literature is
that these interventions should target CSOs instead of individual
citizens, and ensure that the intervention enhances the ability of the
CSOs to overcome their collective action problem (Mattoni and Odilla
2021; Odilla 2023).

The present paper provides evidence supporting this argument. We
reassess the evidence provided by Freire, Galdino, and Mignozzetti
(2020), considering a different intervention than the one they
evaluated. They provided experimental evidence that a mobile app (Tá de
Pé) made available to citizens did not improve the delivery of school
and nursery construction works. Here, we assess the efficacy of a sister
project, called ``Obra Transparente,'' also developed by the Brazilian
NGO Transparência Brasil. This project targeted 21 local Civil Society
Organizations in 21 municipalities in the South and Southeast regions of
Brazil, with the aim of improving access to education by addressing
mismanagement and delays in the construction of public schools and
nurseries. Both projects were developed in parallel, with similar goals
but achieved strikingly different results.

The argument in the present study is that not all types of interventions
that promote citizens' voice are the same. We argue that we need to
differentiate between information-led tactical interventions and active
mobilization and monitoring by Civil Society Organizations. By their
very nature, established CSOs are embedded in society and politics and,
as such, can have teeth of their own (Gong and Xiao 2017; Joshi and
Houtzager 2012). It may be only that we need mobilization by organized
civil society, in a structured way, for civil society voices to be heard
with ``teeth''. Some studies have suggested that a key feature of
successful change in public policies is the pressure of collectively
organized agents (Mattoni and Odilla 2021; Odilla 2023). Collective
action is a hard problem to be overcome and creating CSOs is one way to
achieve this. We argue in this paper that a missing features of the
tactical interventions assessed in most of the literature is that they
should target CSOs instead of individual citizens, and make sure that
the intervention increases the ability of the CSOs to attract and
overcome their collective action problem.

The ``Obra Transparente'' project consisted of developing a network of
local CSOs to share experiences and information on monitoring the
construction works, providing a technical support system for monitoring
the process of constructing public schools and nurseries, and training
on monitoring biddings, contracts, and constructions of public schools
and nurseries.

Our findings emphasize that it is the sustained and concerted efforts of
organized civil society, such as repeated meetings with municipal
officials, persistent follow-ups, and on-site visits, that lead to more
significant outcomes, in contrast to interventions providing information
and generating sporadic citizen actions.

\hypertarget{the-project-obra-transparente}{%
\subsection{The project Obra
Transparente}\label{the-project-obra-transparente}}

The project was initiated in May 2017 by Transparência Brasil (TB) to
address delays in school and nursery construction projects funded by the
federal government and carried out by municipal governments. TB
partnered with Observatório Social do Brasil, an NGO coordinating a
network of local (municipality-based) NGOs, to develop and disseminate a
standardized methodology for local government oversight. The Obra
Transparente project concluded on June 30, 2019.

During the project, TB monitored construction plans, resource
allocation, procurement processes, and work execution. They also tracked
the monitoring and accountability procedures conducted by government
bodies to ensure adherence to technical and legal guidelines.

Transparência Brasil collaborated with the Controladoria Geral da União,
a federal government agency responsible for overseeing federal public
spending, to develop training materials and provide courses to a network
of local CSOs. Courses reached 270 participants, with a mix of online
and in-person modules over several weeks along the duration of the
project. These modules covered topics related to public bidding,
construction monitoring, and contracting procedures.

\hypertarget{research-design-and-methods}{%
\subsection{Research Design and
Methods}\label{research-design-and-methods}}

We employ an integrated inference approach (Humphreys and Jacobs 2023)
to analyze qualitative data within the framework of causal queries. As
this approach is relatively innovative within qualitative research,
we'll provide a comprehensive explanation.

The methodology outlined by (Humphreys and Jacobs 2023) mandates the
creation of a causal model encoded in Directed Acyclic Graphs (DAGs), a
framework developed by Pearl (2009). Bayesian inference, recently
promoted in qualitative data analysis (Fairfield and Charman 2022),
serves as the foundation for deriving quantitative estimates for causal
queries constructed using the proposed DAG.

In our current study, we investigate the impact of Civil Society
Organization (CSO) monitoring on two aspects: the completion of
construction works and the rectification of issues and irregularities
found in construction facilities. Formally, the first research question
relates to a causal relationship between monitoring (M) and completion
(C), indicated by the DAG (M -\textgreater{} C). The second research
query delves into whether monitoring (M) causes the fixing of issues
(F), or the (DAG M -\textgreater{} F). To accommodate the qualitative
nature of our variables, we treat them as binary.

Within this approach, a given DAG allows us to pose causal queries,
which can yield estimands like the average treatment effect (ATE) or
types, which we explain below. Types are the focal estimands in our
study.

This approach hinges on the notion that each case can belong to a
specific type, denoting different causal relationships. In our current
investigation, let's focus on the (M -\textgreater{} C) DAG. To clarify
what a type is, we'll employ the Potential Outcomes (PO) notation (Rubin
2005; Imbens and Rubin 2015). Within the PO framework, every case has
two potential outcomes: one when the treatment (in this case, CSO
monitoring, M) is absent and another when the treatment is present. In
simplified notation, if a city only completes construction works in the
absence of monitoring, we express this in PO notation as C(1) = 0. If
there is no monitoring (M = 0) and no completion, then we write C(0) =
0. Conversely, if there is monitoring (M = 1) and completion, we write
C(1) = 1, and C(0) = 1 if there is completion without monitoring.

We can categorize a city into one of four types based on what happens
when they do or do not receive the treatment (monitoring): Adverse (type
``a''): A city that completes construction only if it doesn't receive
the treatment (monitoring). Beneficial (type ``b''): A city that
completes construction only if there is monitoring. Chronic (type
``c''): Cities that never complete construction, regardless of
monitoring. Destined (type ``d''): Cities that always complete
construction, regardless of monitoring. The PO notation is instrumental
here, as we can rewrite the types based on the following notation: Type
``a'' is one in which the potential outcome is 0 if M = 1 and 1 if M =
0. Therefore, we express it as \(\theta_{10}\), where the first
subscript denotes the outcome when the treatment is 0, and the second
denotes the outcome when the treatment is 1. We employ analogous
notation for all other types: \(b = \theta_{01}\), \(c = \theta_{00}\),
and \(d = \theta_{11}\).

When analyzing a case study of what happened in a given municipality, we
can use the data to answer, for example, the following causal query:
what is the probability that the city if type ``b'', i.e, benefits from
the monitoring and would have not completed the construction work if not
for the monitoring of the local CSO?

The advantage of this approach is that all assumptions and restrictions
imposed to identify the parameters of interest are transparent. It also
allows the user to update posterior probabilities based on qualitative
data. One reason for this is that with Bayesian inference, we do not
need to rely on asymptotic theory to make inferences. Thus, there is no
need for a large-n study. Formally speaking, one can make causal queries
without any data, based only on prior probabilities for the parameters.
Obviously the answer to such causal queries will be vague enough to be
almost useless. But the point is that, once you have a prior and a
causal model, encoded in a DAG, you can answer causal queries. As one
collects data (any data, be it qualitative or quantitative), one can
make assumptions about the nature of the data (such as the Data
Generating Process), and then update the model and provide new answers
to the causal queries, which hopefully will be informative enough to be
useful for the task at hand. And even when the model or hypothesis is
not completely identified, it can be partially identified (Manski 2003),
which is still a useful answer to causal queries.

\hypertarget{data}{%
\subsection{Data}\label{data}}

The evidence gathered is primarily based on documented reports collected
by Transparência Brasil and produced by the local OSBs over the course
of the project. We also use evidence collected on semi-structured
interviews and an online survey with representatives of the 21 CSOs
conducted by an independent evaluator of the project. For the
quantitative analysis, we use administrative data about construction
works data by municipalities and socio-demographic data on
municipalities. The questions asked in the survey as well the aggregate
answered are presented in annex B.

\hypertarget{case-studies}{%
\subsection{Case Studies}\label{case-studies}}

\hypertarget{delivery-of-the-construction-works}{%
\subsubsection{Delivery of the construction
works}\label{delivery-of-the-construction-works}}

In our first case study, we highlight an impressive achievement in
ensuring the completion of construction work. In Taubaté, four
construction projects were underway, which the local OSB began
monitoring as part of Obra Transparente. By the project's conclusion,
their efforts played a crucial role in facilitating the successful
delivery of these schools and nurseries to the city, ensuring they were
ready for use in the field of education

To put in perspective what was accomplished regarding the completion of
all four projects, by the end of the project, of all 135 construction
works planned for all 21 municipalities, only 25 were effectively
delivered, a 18.5\% delivery rate. In Taubaté, before the monitoring by
the local CSO, the delivery rate was zero. After that, the delivery rate
was 100\% and one of the construction work was a previously paralyzed
construction. According to a report by Transparência Brasil (Coelho,
Galdino, and Sakai 2021), out of 771 stalled construction projects that
were eventually completed, over 50\% of them took more than two years to
be delivered and 25\% took three years or more. As we can see, the odds
were against finishing the construction works.

The local CSO monitoring the construction work in Taubaté did one of the
best jobs among all 21 municipalities covered in the project. In fact,
it was mostly the work of a single volunteer with support from the local
OSB and Transparência Brazil throughout the project. She monitored four
ongoing construction works on a periodic basis, she was very
knowledgeable and insistent in demanding response from contractors and
the local government. To give a single example of the quality of the
social monitoring going on, in an email sent on 17th January of 2018 to
the local government, the local OSB volunteer wrote: ``Fazendinha
Daycare: a. construction log verified. There have been a few more
construction workers on the site since December 15, 2017, but not
enough, especially considering the extensive amount of work to be
completed by the new delivery date in April 2018. b. (\ldots). c.~Lack
of roof sealing, leading to water entering the classrooms (a situation
that already existed on November 14, 2017)\ldots{}'' . She was by far
the most knowledgeable and zealous volunteer of the whole project.

All of the construction works were effectively finished and delivered by
the end of the project, although overdue. One of them, a resumption of a
then halted construction work, was delivered 2 months overdue. The
remaining three were further delayed, but no amendments were made on
prices, only deadlines.

From a causal perspective, we are interested in what is called
case-level causal attribution (Humphreys and Jacobs 2023), i.e, did the
monitoring cause the delivery of the construction works? In other words,
we are conditioning on the fact that M = 1 and C = 1, which constrains
the possible types for Taubaté to be \(\theta^C_{1,1}\) ou
\(\theta^C_{0,1}\). Assuming a simple model in which CSO monitoring
causes the completion of works (M -\textgreater{} C), and with the eight
within-case instances, we do the causal query. Our stimand of interest
is the probability that Taubaté is a type \(\theta^D_{10}\).

Assuming uniform priors, the probability that Taubaté is a type
\(\theta^D_{10}\), i.e, it is a city in which the treatment has a
positive effect is 65\%, with 95\% credibility interval around {[}23\%,
95\%{]}. As we can see, a wide interval, due to the low sample size, and
yet, a positive effect nonetheless with 95\% of certainty.

What are the assumptions behind such a result? The main one is that we
modeled the monitoring (the treatment) as-if random, which is obviously
not true. The mere passage of time causes an increased probability of
delivery of a construction work. We can also imagine other possible
confounds. The way to model it in a DAG is to add an arrow from M to C
and vice-versa (M \textless-\textgreater{} C). Now, the estimand of
interest depends on the probability of the treatment (monitoring)
causing also the outcome (Average Treatment Effect, ATE). We will not
detail the derivation of the new stimand here (you can consult annex B),
but the idea is similar to before: we are interested in the type of city
in which there is a causal effect of the monitoring on the completion of
a construction works of school and/or nurseries. The mean probability of
Taubaté being of this type, assuming the model is true is 43\%, with
95\% credibility intervals around {[}9\%, 80\%{]}. The interval is wider
because we added the uncertainty of confounding. Nonetheless, it is
unlikely that the positive effect found in the observational data is
driven only due to the confounding variables.

\hypertarget{preventing-problems-before-projects-started}{%
\subsubsection{Preventing problems before projects
started}\label{preventing-problems-before-projects-started}}

The local OSB monitoring in Auraucária story is one when oversight
arrives at the right moment to prevent a costly government procurement.

In 2017, the Brazilian municipality of Araucária conducted bidding
processes seeking to hire companies to build three nurseries. The
project sent technical experts to the 3 planned construction sites,
together with volunteers from the local Observatório, to check whether
plans were consistent with the physical conditions observed on site. The
experts' assessment was that high-cost contention walls included in the
construction plans were unnecessary or could be replaced by low-cost
solutions through adjustments. Originally, the contention walls would
cost R\$ 1,577,338.57 (US\$ 419,504.9, at the nominal exchange rate in
2019). The analysis was first directed to the local administration,
suggesting changes to the construction plans, which were rejected. The
bidding processes went on as initially planned. Subsequently, the
experts' findings were submitted to the Brazilian Supreme Audit
Institution, Tribunal de Contas da União (TCU), by Transparência Brasil
and Observatório Social do Brazil (see annex 1). TCU recommended that
the bidding process should be redone, and construction plans were to be
redesigned, with more cost-effective solutions for the contention walls.
In the new bidding process, their cost plummeted to R\$ 416,883.17 (USD
110,873.2), a reduction of USD 308,631.7 --- or 74\% less. To give a
sense of the impact on the city budget, the amount saved on these three
constructions alone represents about 3\% of all capital investments of
Araucária in 2018. In short, in a single municipality, the project
resulted in savings higher than the cost of the entire project funded by
the grantee UNDEF (US\$ 220,000.00).

\hypertarget{fixing-defects-in-ongoing-projects}{%
\subsubsection{Fixing defects in ongoing
projects}\label{fixing-defects-in-ongoing-projects}}

Another impact of the project was the correction of the irregularities
and construction failures, improving the quality of the delivered
construction work. During their monitoring activities, the OSBs found
defects and issues, promptly reported them to municipality
administration, which then acted in order to fix them. The project
partners ended up substituting government inspectors and making up for
their shortcomings in ensuring contract compliance through supervision.
This was the case in both the four constructions in Taubaté (SP) and
three that were going on in Foz do Iguaçu (SC). In both cases volunteers
performed monitoring activities very frequently, which allowed for a
thorough investigation and revealed a series of defects in projects and
execution.

\hypertarget{fixing-defects-in-finished-projects}{%
\subsubsection{Fixing defects in finished
projects}\label{fixing-defects-in-finished-projects}}

In Palhoça (SC), is an example when oversight came too late. One of the
daycares completed before the monitoring period started was delivered
with a wall of perforated elements without proper grouting, with pieces
that were already coming loose. The daycare was put into operation
without addressing the issue, which led the staff to isolate the area to
minimize the risk of injuries to the children. In other words, part of
the building cannot be used due to a failure for which a solution should
have already been demanded from the responsible company.

In Goioerê, the story had a more happy ending. Volunteers visited an
already delivered school building and noticed improperly installed
windows, rendering them non-functional. A series of finishes were also
lacking, such as grab bars and handles in accessible restrooms for
persons with disabilities or reduced mobility. These shortcomings
compromised adequate use of the building and exposed its users - mainly
children - to potentially dangerous situations. These flaws were mostly
corrected by the contractor after being reported by the local
Observatório.

\hypertarget{quantitative-observational-evidence}{%
\subsection{Quantitative observational
evidence}\label{quantitative-observational-evidence}}

We have already observed qualitative evidence indicating that local OSB
monitoring played a crucial role in expediting the completion of
construction projects. Now, we present quantitative evidence regarding
the impact of this monitoring on delivery rates.

To do so, we collected data on school and nursery construction projects
across the entire country, encompassing over two thousand municipalities
and more than 14 thousand construction projects, spanning all phases
from planning to execution and completion. The
(\textbf{ref?})(tab:table1) below provides descriptive statistics for
these construction projects.

When comparing ``treated'' and ``untreated'' units, there is difference
between the groups, as show below by the percentage of construction
works within both groups before the project started.

\setlength{\LTpost}{0mm}
\begin{longtable}{rrr}
\caption*{
{\large Percentage of finished construction works before the project} \\ 
{\small Situation in August 2017 and August 2019}
} \\ 
\toprule
\# of participating cities & \% (Before Project) & \% (After Project) \\ 
\midrule\addlinespace[2.5pt]
0 & $49\%$ & $59\%$ \\ 
1 & $29\%$ & $42\%$ \\ 
\bottomrule
\end{longtable}
\begin{minipage}{\linewidth}
Construction projects in states SP, MG, SC, PR and RS.\\
\end{minipage}

Our primary objective is to compare the project's effects on the rate of
delivery. To estimate the Average Treatment Effect on the Treated (ATT),
we employed a matching strategy (Ho et al. 2007; Rubin 1973) with a
Bayesian logistic regression to create the propensity scores. Balance
was successful after matching. We also applied the same procedure to
pre-project data to ensure that there were no substantial differences
between the treated and control groups.

We estimate all models using the a varying-intercept regression equation
(Gelman and Hill 2006):

\[
Y_i = \alpha + \beta_1 T_i + \gamma X_i + \theta Z_i + \epsilon_i
\]

Where \(i\) indexes the construction units, \(Y_i\) is \(1\) if a
construction work is finished and \(0\) otherwise. \(\alpha\) represents
the intercept, \(\beta\) denotes the average treatment effect on the
treated (ATT), and \(T_i\) is a binary treatment indicator (whether the
municipality is part of Obra Transparent or not). Symbol \(\gamma\) is a
vector of varying intercepts, \(X_i\) is a matrix of random effects for
Brazilian municipalities, \(\theta\) is a vector of control variables,
and \(Z_i\) is an array of control variables for case \(i\). The error
term is represented by \(\epsilon_i\). We employed a Bayesian model with
rstanarm (Goodrich et al. 2023) with default weakly informative priors
on the parameters (Gelman et al. 2008).

The table below presents estimates for the causal effect on the treated
units regarding the change in completion rates before and after the
project.

Table (insert ref number) summarizes the main results of the regression.
The first row presents the results for the regression before the project
started, where we expect the effect of the project to be zero, since it
didn't exist. The second row shows the effect after the end of the
project, when we expect the impact to be positive. The first column is
an estimate of the posterior mean of the impact of the project and the
second one is the posterior standard deviation of the estimate. We also
present posterior median and \(2.5%
\) and \(97.5%
\) quantiles, in order to quantify the uncertainty around the estimates
with a 95\% credibility interval.

The hypothesis are corroborated, and we did not find an effect before
the starting of the project, but found an effect aftet the end of the
project. On average, a city that received the project Obra Transparente
increases its delivery rate by almost 10\%. These results support our
argument and the qualitative evidence we presented that sustained and
costly bottom-up accountability improve public policies.

\setlength{\LTpost}{0mm}
\begin{longtable}{lrrrrr}
\caption*{
{\large Effect of local OSB social monitoring on construction delivery rates} \\ 
{\small Posterior probabilities}
} \\ 
\toprule
Variables & Mean & Sd & Median & 2.5\% CI & 97.5\% CI \\ 
\midrule\addlinespace[2.5pt]
ATT before project & $0.025$ & $0.042$ & $0.026$ & $-0.059$ & $0.108$ \\ 
ATT After project & $0.099$ & $0.049$ & $0.099$ & $0.001$ & $0.194$ \\ 
\bottomrule
\end{longtable}
\begin{minipage}{\linewidth}
452 observations for the pre-treatment period and 320 observations for the post-treatment period, resulting from the matching procedure, and encompassing municipalities in the same states as municipalities of Obra Transparente\\
\end{minipage}

\hypertarget{discussion-of-results}{%
\subsection{Discussion of Results}\label{discussion-of-results}}

One of the reasons identified for some of the problems with the
construction works is lack of qualified personnel from the government to
monitor the contractor work. As a result, the local government does not
properly oversight the work of contractors. And this is a task that
requires technical expertise and is time consuming, involving going
multiple times to the construction site, comparing the construction log
with the schedule to assess if there is potential delay and what is
specified in the contract, to assess if the materials used are correct,
if everything demanded is being done etc.

The fact that an intervention like Tá de Pé (Freire, Galdino, and
Mignozzetti 2020), which provided information to citizens about
construction work, but did not produce a meaningful impact does not mean
that bottom up accountability does not have any effect. The evidence we
gathered here, assessed with multiple methods, clearly shows that costly
and consistent oversight by civil society can produce results that are
quite impactful, be it increased rate of project completion, improved
quality of the construction works or decreasing in governmental
spending.

The detection of evidence of procurement fraud cases also shows local
administrations are not diligent enough in conducting procurement
procedures, and the local partners of TB contributed to addressing that
gap as well. Moreover, as federal audit agencies lack the capacity to
adequately oversee and analyze the immense volume of contracting
procedures conducted at the local level, the social monitoring resulted
in the identification of high-risk cases that otherwise might not have
been assessed by the competent bodies.

\hypertarget{concluding-remarks}{%
\subsection{Concluding Remarks}\label{concluding-remarks}}

Accountability, as a fundamental element in democratic systems, plays a
pivotal role in ensuring the efficient provision of public services.
Recent studies have brought to light the concept of ``bottom-up
accountability,'' where citizens receive information about public
policies and utilize it to exert social control.

While tactical interventions aimed at providing information to citizens
have shown limited impact, it would be a mistake to dismiss the
potential of citizen voices in governance entirely. Our research has
differentiated between information-led tactical interventions and active
mobilization and monitoring by Civil Society Organizations (CSOs). We
argue that organized CSOs, deeply rooted in society and politics, hold
the capacity to bring about substantial policy change.

This paper provides evidence in support of this argument. By assessing
the effectiveness of the ``Obra Transparente'' project developed by
Transparência Brasil, we illustrate the significant role that organized
CSOs can play in shaping government behavior and improving public
service delivery. Unlike interventions solely focused on providing
information, this project aimed at creating a network of local CSOs,
offering technical support, and providing training on monitoring public
school and nursery construction projects.

Our findings underscore the importance of sustained, concerted efforts
by organized civil society, including direct engagement with municipal
officials, persistent follow-ups, and on-site visits. These endeavors
yield more substantial outcomes compared to interventions that merely
deliver information and generate sporadic citizen actions.

In a world where accountability mechanisms continue to evolve, the role
of organized civil society remains paramount. The ability to foster
collective action, create informed oversight, and influence policy
change is an essential aspect of accountable and effective democratic
governance. The journey toward enhancing bottom-up accountability is
ongoing, and our research contributes to a more comprehensive
understanding of how it can be achieved through the mobilization of
organized civil society.

Sample figure:

\begin{figure}
Figure here.

\caption{Caption for figure below.}
\begin{figurenotes}
Figure notes without optional leadin.
\end{figurenotes}
\begin{figurenotes}[Source]
Figure notes with optional leadin (Source, in this case).
\end{figurenotes}
\end{figure}

Sample table:

\begin{table}
\caption{Caption for table above.}

\begin{tabular}{lll}
& Heading 1 & Heading 2 \\
Row 1 & 1 & 2 \\
Row 2 & 3 & 4%
\end{tabular}
\begin{tablenotes}
Table notes environment without optional leadin.
\end{tablenotes}
\begin{tablenotes}[Source]
Table notes environment with optional leadin (Source, in this case).
\end{tablenotes}
\end{table}

\hypertarget{references}{%
\subsection{References}\label{references}}

\bibliographystyle{aea}
\bibliography{references}

\% The appendix command is issued once, prior to all appendices, if any.
\appendix

\section{Mathematical Appendix}

\hypertarget{refs}{}
\begin{CSLReferences}{1}{0}
\leavevmode\vadjust pre{\hypertarget{ref-ankamah_2019}{}}%
Ankamah, Samuel Siebie. 2019. {``Why Do {`Teeth'} Need {`Voice'}? {The}
Case of Anti-Corruption Agencies in Three {Australian} States.''}
\emph{Australian Journal of Public Administration} 78 (4): 481--96.

\leavevmode\vadjust pre{\hypertarget{ref-bertelli_2020}{}}%
Bertelli, Anthony M., and Gregg G. Van Ryzin. 2020. {``Heuristics and
Political Accountability in Complex Governance: {An} Experimental
Test.''} \emph{Research \& Politics} 7 (3): 2053168020950080.
\url{https://doi.org/10.1177/2053168020950080}.

\leavevmode\vadjust pre{\hypertarget{ref-besley_2003}{}}%
Besley, Timothy, and Maitreesh Ghatak. 2003. {``Incentives, Choice, and
Accountability in the Provision of Public Services.''} \emph{Oxford
Review of Economic Policy} 19 (2): 235--49.

\leavevmode\vadjust pre{\hypertarget{ref-bjorkman_2009}{}}%
Björkman, Martina, and Jakob Svensson. 2009. {``Power to the People:
Evidence from a Randomized Field Experiment on Community-Based
Monitoring in {Uganda}.''} \emph{The Quarterly Journal of Economics} 124
(2): 735--69.

\leavevmode\vadjust pre{\hypertarget{ref-cameron_2004}{}}%
Cameron, Wayne. 2004. {``Public Accountability: {Effectiveness}, Equity,
Ethics.''} \emph{Australian Journal of Public Administration} 63 (4):
59--67.

\leavevmode\vadjust pre{\hypertarget{ref-coelho_etal2021}{}}%
Coelho, Jonas, Manoel Galdino, and Juliana Sakai. 2021.
{``{Levantamento: Obras federais paralisadas de creches e escolas com
maiores chances de serem concluídas}.''}

\leavevmode\vadjust pre{\hypertarget{ref-fairfield_2022}{}}%
Fairfield, Tasha, and Andrew E. Charman. 2022. \emph{Social Inquiry and
{Bayesian} Inference: {Rethinking} Qualitative Research}. {Cambridge
University Press}.

\leavevmode\vadjust pre{\hypertarget{ref-fox_2015}{}}%
Fox, Jonathan A. 2015. {``Social Accountability: What Does the Evidence
Really Say?''} \emph{World Development} 72: 346--61.

\leavevmode\vadjust pre{\hypertarget{ref-freire_etal2020}{}}%
Freire, Danilo, Manoel Galdino, and Umberto Mignozzetti. 2020.
{``Bottom-up Accountability and Public Service Provision: {Evidence}
from a Field Experiment in {Brazil}.''} \emph{Research \& Politics} 7
(2): 1--8. \url{https://doi.org/10.1177/2053168020914444}.

\leavevmode\vadjust pre{\hypertarget{ref-gelman_2006}{}}%
Gelman, Andrew, and Jennifer Hill. 2006. \emph{Data Analysis Using
Regression and Multilevel/Hierarchical Models}. {Cambridge university
press}.

\leavevmode\vadjust pre{\hypertarget{ref-gelman_etal2008}{}}%
Gelman, Andrew, Aleks Jakulin, Maria Grazia Pittau, and Yu-Sung Su.
2008. {``A Weakly Informative Default Prior Distribution for Logistic
and Other Regression Models.''}

\leavevmode\vadjust pre{\hypertarget{ref-gong_2017}{}}%
Gong, Ting, and Hanyu Xiao. 2017. {``Socially Embedded Anti-Corruption
Governance: {Evidence} from {Hong Kong}.''} \emph{Public Administration
and Development} 37 (3): 176--90.

\leavevmode\vadjust pre{\hypertarget{ref-goodrich_etal2023}{}}%
Goodrich, Ben, Jonah Gabry, Imad Ali, and Sam Brilleman. 2023.
{``Rstanarm: {Bayesian} Applied Regression Modeling via {Stan}.''}
\emph{R Package Version} 2 (26.1).

\leavevmode\vadjust pre{\hypertarget{ref-ho_etal2007}{}}%
Ho, Daniel E., Kosuke Imai, Gary King, and Elizabeth A. Stuart. 2007.
{``Matching as Nonparametric Preprocessing for Reducing Model Dependence
in Parametric Causal Inference.''} \emph{Political Analysis} 15 (3):
199--236.

\leavevmode\vadjust pre{\hypertarget{ref-humphreys_2023}{}}%
Humphreys, Macartan, and Alan M. Jacobs. 2023. \emph{{INTEGRATED
INFERENCES}: {Causal Models} for {Qualitative} and {Mixed-method
Research}}. {CAMBRIDGE University Press}.

\leavevmode\vadjust pre{\hypertarget{ref-imbens_2015}{}}%
Imbens, Guido W., and Donald B. Rubin. 2015. \emph{Causal Inference in
Statistics, Social, and Biomedical Sciences}. {Cambridge University
Press}.

\leavevmode\vadjust pre{\hypertarget{ref-joshi_2012}{}}%
Joshi, Anuradha, and Peter P. Houtzager. 2012. {``Widgets or Watchdogs?
{Conceptual} Explorations in Social Accountability.''} \emph{Public
Management Review} 14 (2): 145--62.

\leavevmode\vadjust pre{\hypertarget{ref-kenney_2003}{}}%
Kenney, Charles D. 2003. {``Horizontal Accountability: Concepts and
Conflicts.''} In \emph{Democratic {Accountability} in {Latin America}},
55--76. {Scott Mainwaring \& Christopher Welna}.

\leavevmode\vadjust pre{\hypertarget{ref-manski_2003}{}}%
Manski, Charles F. 2003. \emph{Partial Identification of Probability
Distributions}. Vol. 5. {Springer}.

\leavevmode\vadjust pre{\hypertarget{ref-mattoni_2021}{}}%
Mattoni, Alice, and Fernanda Odilla. 2021. {``Digital {Media},
{Activism}, and {Social Movements}' {Outcomes} in the {Policy Arena}.
{The Case} of {Two Anti-Corruption Mobilizations} in {Brazil}.''}
\emph{Partecipazione e Conflitto} 14 (3): 1127--50.

\leavevmode\vadjust pre{\hypertarget{ref-mulgan_2000}{}}%
Mulgan, Richard. 2000. {``{`{Accountability}'}: An Ever-Expanding
Concept?''} \emph{Public Administration} 78 (3): 555--73.

\leavevmode\vadjust pre{\hypertarget{ref-odonnell_1998}{}}%
O'donnell, Guillermo. 1998. {``Horizontal Accountability in New
Democracies.''} \emph{J. Democracy} 9: 112.

\leavevmode\vadjust pre{\hypertarget{ref-odonnell_2003}{}}%
---------. 2003. {``Horizontal Accountability: {The} Legal
{Institutionalization} of {Mistrust}.''} In \emph{Democratic
{Accountability} in {Latin America}}, 34:34--54. {Scott Mainwaring \&
Christopher Welna}.

\leavevmode\vadjust pre{\hypertarget{ref-oloughlin_1990}{}}%
O'Loughlin, Michael G. 1990. {``What Is Bureaucratic Accountability and
How Can We Measure It?''} \emph{Administration \& Society} 22 (3):
275--302.

\leavevmode\vadjust pre{\hypertarget{ref-odilla_2023}{}}%
Odilla, Fernanda. 2023. {``Bots Against Corruption: {Exploring} the
Benefits and Limitations of {AI-based} Anti-Corruption Technology.''}
\emph{Crime, Law and Social Change}, 1--44.

\leavevmode\vadjust pre{\hypertarget{ref-paul_1992}{}}%
Paul, Samuel. 1992. {``Accountability in Public Services: Exit, Voice
and Control.''} \emph{World Development} 20 (7): 1047--60.

\leavevmode\vadjust pre{\hypertarget{ref-raffler_etal2019}{}}%
Raffler, Pia, Daniel N. Posner, and Doug Parkerson. 2019. {``The
Weakness of Bottom-up Accountability: Experimental Evidence from the
{Ugandan} Health Sector.''} \emph{Los Angeles: Innovations for Poverty
Action}.

\leavevmode\vadjust pre{\hypertarget{ref-relly_2012}{}}%
Relly, Jeannine E. 2012. {``Examining a Model of Vertical
Accountability: {A} Cross-National Study of the Influence of Information
Access on the Control of Corruption.''} \emph{Government Information
Quarterly} 29 (3): 335--45.

\leavevmode\vadjust pre{\hypertarget{ref-rubin_1973}{}}%
Rubin, Donald B. 1973. {``Matching to Remove Bias in Observational
Studies.''} \emph{Biometrics}, 159--83.

\leavevmode\vadjust pre{\hypertarget{ref-rubin_2005}{}}%
---------. 2005. {``Causal Inference Using Potential Outcomes: {Design},
Modeling, Decisions.''} \emph{Journal of the American Statistical
Association} 100 (469): 322--31.

\leavevmode\vadjust pre{\hypertarget{ref-serra_2012}{}}%
Serra, Danila. 2012. {``Combining Top-down and Bottom-up Accountability:
Evidence from a Bribery Experiment.''} \emph{The Journal of Law,
Economics, \& Organization} 28 (3): 569--87.

\end{CSLReferences}


\end{document}
